\documentclass{article}

\usepackage{amsmath}
\usepackage{tikz}

\begin{document}

\title{Unlucky 13: A Negative Result in Digital Physics}
\author{Moshe Looks}


\maketitle


The distance $d_2(p, q)$ between points in two-dimensional
Euclidean space is $\sqrt{(p_1 - q_1)^2 + (p_2 - q_2)^2}$. The former can only ever
approximate the latter, because $\sqrt{2}$ is irrational. How well can we approximate it
using a graph of $N$ vertices?


The distance $d_G(u, v)$ between vertices in an undirected graph is the number of edges in a
shortest path between $u$ and $v$. The distance $d_2(p, q)$ between points Euclidean space
is the $L_2$ of the line segment between $u$ and $v$. In two or more dimensions $d_G$ will
always be unequal to $d_2$ because of the incommensurability of the side and diagonal of the
square. But $d_G$ \emph{can} approximate $d_2$. How well can $d_G$ approximate $d_2$, using
a graph with $N$ vertices?


$\sqrt{(p_1 - q_1)^2 + (p_2 - q_2)^2}$. The former can only ever
approximate the latter, because $\sqrt{2}$ is irrational. How well can we approximate it
using a graph of $N$ vertices?

This is an interesting question, but difficult to answer in full generality. Let's consider
the special case of a graph with $N = n^2$ vertices, arranged in a square grid. Connecting
each vertex with its horizontal and vertical neighbors gives us Manhattan ($L_1$) distances,
which are too large. Connect each vertex also with its diagonal neighbors gives us chessboard
($L_\infty$) distances, which are too small.

What if we connect only \emph{some} vertices along their diagonals? For example:

\input{fig1.tex}

This graph has an interesting property; for every Pythagorean triple\footnote{Triples of
natural numbers $(a, b, c)$ s.t. $a^2 + b^2 = c^2$; eg. $(3, 4, 5)$, $(12, 5, 13)$, \&c.}
$(a, b, c)$ and every pair of vertices $(p, q)$ placed respectively at $(x, y)$ and $(x + a,
y + b)$, $d(p, q) = c$. Let's call an undirected graph whose vertices can be placed on an $n$
by $n$ square grid an order-$n$ Eugrid (Euclidean grid).

Because Pythagorean triples are dense in the rational numbers, a Eugrid of sufficiently high
order would in a certain sense capture the structure of Euclidean space, insofar as this is
possible at all with a finite square grid. But do high-order Eugrids exist?\footnote{Spoiler:
they do not.}

We can naively consider $2^{(n-1)^2}$ possibilities for an order-$n$ Eugrid by putting
undirected graphs in correspondence with bit matrices where $1$s correspond to diagonal edges
in the corresponding undirected graphs.\footnote{We need only consider diagonals with a
single fixed orientation ($/$ and not $\backslash$) because of the symmetry of the
space.}. With this convention, the order-$5$ Eugrid exhbited above corresponds to the $4 x 4$
Eugridean bit matrix

\begin{equation*}
\begin{matrix}
  0 & 0 & 0 & 0 \\
  0 & 0 & 1 & 0 \\
  0 & 1 & 0 & 1 \\
  0 & 0 & 1 & 0
\end{matrix}
\end{equation*}

For $n=5$ this representation gives us only $2^{16} = 65,536$ cases, and we can brute-force
it to see that there are 10,836 Eugrids of this order; they are rather thick on the ground.

What to do about higher orders? $2^{25} = 33,554,432$ which is already rather large, and
$2^{36} = 68,719,476,736$ and above are quite unpleasant. We can make some progress by
noticing that higher-order Eugrids must be composed of lower-order ones. In particular, if
matrix $A$ corresponds to an order-$n$ Eugrid, then all submatrices $A_{1:m,1:m}$ correspond
to order-$m$ Eugrids.

This naturally suggests a partition of the full search space having $(n-1)^2$ dimensions into
$(n-1)$ ``layers'', like so:

\input{fig2.tex}

Possible diagonals for $n$th layer correspond to squares numbered $n$ in the figure. So
rather than constructing the entire space in one go, we only ever worry about a particular
layer, and assume that all smaller layers are valid (i.e. correspond to Eugrids). Of course,
we may have to backtrack, because not all lower-order Eugrids compose to form higher-order
Eugrids.

We have more work to do in order to make the search tractable. The first step is to move away
from brute-force enumeration when considering diagonals for layer $n+1$ given layer $n$
already contains a Eugrid.\footnote{The $n=0$ case corresponds to a layer only a single
search space variable, so we don't mind enumerating over it.} The basic idea here is that
every region of the space corresponding to a Pythagorean triple $(a, b, c)$ with lower-left
corner $(x, y)$ corresponds to a set of constraints on the diagonals inside of it, and we
will end up with a Eugrid iff \emph{all} such sets of constraints are satisfied.

What are these constraints, exactly? If all variables corresponding to diagonals in a
particular region have been assigned, then obviously the constraints require the shortest
paths from $(x, y)$ to $(x+a, y+b)$ have length $c$. But we can do better than this and
impose constraints on partially assigned regions as well. For example, no Eugridean matrix
can contain
\begin{equation*}
\begin{matrix}
  1 & * & * \\
  * & 1 & * \\
  * & * & 1
\end{matrix}
\end{equation*}
as a proper submatrix (where ``*'' may be either a 1 or a 0) because if so then it would be
contained within a $3x4$ region\footnote{Corresponding to the Pythagorean triple $(3, 4,
5)$.} with a shortest path for the hypotenuse of length $< 5$ in violation of Eugrideanity.
Likewise $0_{2 x 4}$ is not a submatrix of any Eugridean, because it would lead to a similar
region with hypotenuse of length $> 5$.

Since graph distance equals shortest path length, $d(u, v) = c$ requires both that no path
from $u$ to $v$ be shorter than $c$, \emph{and} that at least one path be no longer than $c$.
The partition of a square grid into layers as we have done dictates that all shortest pathsbetween.

To get tight bounds, recall that Eugridean distance is lower-bounded by $L_\infty$ and
upper-bounded by $L_1$.

For every literal \verb|x| and corresponding vertex $x$, \emph{if} there exists a region
$r = (p, q, c)$ s.t. $d(p, x) + d_1(x, q) == c$, \emph{then} $r$ is unconstrained \emph{and}
\verb|x| is negated.

For every region $r = (p, q, c)$, \emph{if} there exists a vertex $x$ s.t.
$d(p, x) + d_{\inf}(x, q) < c$, \emph{then} $r$ is unconstrained.

All other regions are constrained. For every constrained region $r = (p, q, c)$ at least one
literal \verb|x| corresponding to vertex $x$ that satisfies $d(p, x) + d_{\inf}(x, q) = c$
must be affirmed.

We can thus construct a logical conjunction of clauses where every clauses is either a
negated literal or a disjunction of non-negated literals s.t. the layer is valid iff the the
conjunction is satisfied. Whereas general Boolean satisfiability is a hard problem, formulae
with this special form are easily checkable. We can easily enumerate all valid assignments
using depth-first search.

This leads to a backtracking search procedure for finding Eugrids:

Let $1$ be the active layer.

Generate a satisfying assignment for the active layer. If no satisfying assignment exist, or
if we have already generated all satisfying assignments, backtrack to the previous layer.

Advance to the next layer.


\end{document}
